\documentclass[11pt,letterpaper]{article}

\usepackage[margin=1in]{geometry}
\usepackage[
  colorlinks=true,
  pdfencoding=auto,
  pdfauthor={Dale L. Peterson},
  pdfkeywords={dynamics; control; control moment gyroscope},
  pdftitle={C-1 dynamics model},
  xetex
]{hyperref}
\usepackage{amssymb}
\usepackage{amsmath}

\begin{document}
\title{Simplified CMG dynamics}
\author{Dale L. Peterson\\
        Lit Motors Inc.\\
        \texttt{luke@litmotors.com}}
\date{\today}

\maketitle
\abstract{This document presents the equations of motion for a flywheel with
    roll, gimbal, and spin degrees of freedom. The roll, gimbal, and spin
    axes are assumed mutually perpendicular and to intersect at the system
    mass center.  The inertia of the flywheel enclosure is ignored to highlight
    the effect of the rotating flywheel.  The gimbal axis is actuated with an
    external torque while the roll axis is not actuated directly.}

\section{Introduction}
Consider the inertial frame $A$ with $\hat{a}_z$ pointed down and parallel to
gravity.  The roll frame $B$ is oriented relative to $A$ by a simple rotation
about the $\hat{a}_x$ by roll angle $q_0$. The enclosure fixed frame $C$ is
oriented relative to $B$ by a simple rotation about $\hat{b}_y$ by gimbal angle
$q_1$. The flywheel fixed frame $D$ is oriented relative to the enclosure frame
$C$ by a simple rotation about $\hat{c}_z$ by flywheel angle $q_2$.

The inertia of the flywheel is $I \hat{c}_x \hat{c}_x + I \hat{c}_y \hat{c}_y +
J \hat{c}_z \hat{c}_z$.  This inertia is expressed in the the enclosure fixed
frame $C$ rather than the flywheel fixed $D$ frame; this is possible since the
flywheel is inertially symmetric about its spin axis. The equations of motion
are
\begin{equation}
    \begin{bmatrix}
        (J - I) \sin^2 q_1 + I & 0 & J \sin q_1 \\
        0 & I & 0 \\
        J \sin q_1 & 0 & J
    \end{bmatrix}
    \begin{bmatrix}
        \dot{u}_0 \\ \dot{u}_1 \\ \dot{u}_2
    \end{bmatrix}
    =
    \begin{bmatrix}
        -2 (J - I) \sin q_1 \cos q_1 u_0 u_1  - J \cos q_1 u_1 u_2 \\
        (J - I) \sin q_1 \cos q_1 u_0^2 + J \cos q_1 u_0 u_2 + \tau \\
        -J \cos q_1 u_0 u_1
    \end{bmatrix}
\end{equation}
Solving for $\dot{u}$ yields
\begin{equation}
    \begin{bmatrix}
        \dot{u}_0 \\ \dot{u}_1 \\ \dot{u}_2
    \end{bmatrix}
    =
    \begin{bmatrix}
        (2 - \frac{J}{I}) \tan q_1 u_0 u_1 - \frac{J}{I \cos q_1} u_1 u_2 \\
        (\frac{J}{I} - 1) \sin q_1 \cos q_1 u_0^2 + \frac{J}{I} \cos q_1 u_0 u_2 + \frac{\tau}{I} \\
        \left( (\frac{J}{I} - 1) \sin q_1 \tan q_1 - \frac{1}{\cos q_1} \right) u_0 u_1 + \frac{J}{I} \tan q_1 u_1 u_2
    \end{bmatrix}
\end{equation}
\end{document}
